\documentclass[10pt,a4paper]{article}
\usepackage[utf8x]{inputenc}
\usepackage{ucs}
\usepackage{amsmath}
\usepackage{amsfonts}
\usepackage{amssymb}
\usepackage{makeidx}
\author{Claire Dune}
\title{\LaTeX par la pratique}

\begin{document}
\maketitle
\tableofcontents


\section*{Résumé}
En quelque phrases, résumer la problématique, les hypothèses de travail et exposer une synthèse des contributions de l'article.
\section{Introduction}

Introduire en une demi page le contexte du travail.\\

La section \ref{sec:biblio} est la bibliographie, la section \ref{sec:methodo} présente la méthodologie, la section \ref{sec:resultat} présente les résultats et la section \ref{sec:conclusion} donne les conclusions.



Etude des travaux prééxistants et positionnement.

\section{Méthodologie}
\label{sec:methodo}

Description détaillée des contributions et mise en oeuvre des tests.

\subsection{Aspects théoriques}

Toutes les formules mathématiques et les figures.

\subsection{Protocole expérimental}
Exposé du choix du protocole expérimental.

\section{Bibliographie}
\label{sec:biblio}

\section{Résultats}
\label{sec:resultat}
Détail des résultats.


\section{Conclusion}
\label{sec:conclusion}
Rappeler le contexte et donner une synthèse des résultats en appuyant sur les contributions. Lister les perspectives.
\end{document}